\section{Service descriptions}
The attached files named ``google-services.xml'' and
``yahoo-services.xml'' are examples our service descriptions.

\subsection{Definitions} 
In the following, capitalized names correspond to actual classes in the
code:
\begin{itemize}
    \item a {\em \code{Service}} is an implementation or
      \code{Relation}, whose \code{access()} method performs the an access
      with input and output tuples, hiding all the details about how
      it is actually done.
    \item a {\em \code{ServiceRepository}} is a collection of \code{Service}s
    \item a {\em \code{UsagePolicy}}, is a rule that a service vendor imposes on
      users (e.g. max per-day request, max per-day results, paging,
      monetary cost, ``backoff'' policy, etc.)
    \item an {\em \code{InputMethod}} is a scheme imposed on the user to provide
      input value (e.g. URL params, URL path element, batch inputs, etc.)
    \item an {\em \code{OutputMethod}} is a schema to extract a piece of results
      (that usually comes as a tree) and put it into a
      tuple. Currently, this is expressed as a path.
\end{itemize}

\subsection{Service descriptions structure}
The service description files are structured as follows:
\begin{itemize}
    \item each represent exactly one \code{ServiceRepository}
    \item a \code{ServiceRepository} may define a set of common usage
      policies (shared across all its services)
    \item a \code{ServiceRepository} may define a set of common input methods
      (shared across all its services)
\end{itemize}

Then follow a collection of Services, each one is structured as
follows:
\begin{itemize}
    \item a \code{Service} has a name (the relation name), a protocol, some
      base URI, link to the online documentation, and a path fragment
      defining how sequential results are delimited.
    \item a \code{Service} may define specific usage policies and input
      methods.
    \item then comes a list of static inputs and attributes. These may
      be interleaves, but the order is important.
    \item finally, comes a list of access methods, with the usual
      name, type, inputs positions, and (currently fixed) cost.
\end{itemize}

\subsection{Inputs and attributes}
Static inputs are typical inputs that are required for the service to
function, but have little (or nothing) to do with the results content.
They are specified at the level of a single Service as they may be
different from one \code{Service} to another in a single \code{ServiceRepository}.
They are used during an access to form the request, but they do not
appear as part of the relation's attribute.
Each static input is associated with an \code{InputMethod}, and possibly a
static value.

\code{Attribute}s are the actual attributes of the \code{Service}/\code{Relation}.
In addition to the usual type, that are associated with an
\code{InputMethod} (if they can be used as inputs), and an output path.
Some attributes may be used as both inputs and outputs, in which case,
they required both input and output methods.
However, sometimes an attribute can only be used as an input, in which
case, it is not always possible to get a value for it from the service
response.
For instance, the YahooPlaces has a ``keyword'' attribute. If used, it
is possible to place the input value in the output tuple.
When it is not used, say the woeid was used as input instead, the
attributes will have no value in the resulting tuples.

The limited access method's input positions corresponds to attributes
only (i.e. static inputs are ignored).

\subsection{Input and output methods}
The rationale for introducing the input and output method was to avoid
(or a least postpone) the need for defining complex types in Java
class.
It is easy to get caught into defining classes for every complex type
we come across, and I am afraid we would have to maintain a lot of
those if we follow this path.

What I came up with may be perfectible, but I think it will allows us
to cover a lot of cases will sticking to simple type for our relation
attributes.
You will notice that some input methods feature a ``template''
attribute, with values of the form ``.type('{1}')'', such as (in the
google file)

\begin{verbatim}
    <input-method name="locations" type="url-param" template="{1},{2}"  />
\end{verbatim}

When an attribute or static input is associated with such an input
method, it must specify where in the template its value will be
placed. For instance, 

\begin{verbatim}
    <attribute name="latitude"  input-method="locations.1" path="location/lat" ... />
    <attribute name="longitude" input-method="locations.2" path="location/lng" ... /> 
\end{verbatim}

An access with input tuple (``33.5'', ``-142.1'') will form a URL
param ``locations=33.5,-142.1''.

The outpath extract scheme is primitive for now, i.e. does now always
very complex expression. But it can already let us ``simplify''
complex type, as in the latitude/longitude case above.

\subsection{Usage policies}

One can define usage policies for various kind of cases. Usage
policies are class than implement either or both of the
\code{PreAccessUsagePolicy} and \code{PostAccessUsagePolicy}.

The actual behaviour of this method is proper to each implementation.
A usage policy may for instance,
\begin{inparaenum}[(i)]
  \item put the current thread to sleep for a given period of time if
    the service requires the client to wait before sending a request,
  \item throw a \code{UsagePolicyViolationException} if the policy
    prevent the access complitaly (\eg a quote was reached),
  \item modify the service request building process (\eg if paging is
    required to obtained a complete results for a single access),
\end{inparaenum}
etc.

The \code{UsagePolicy} classes follow an event based architecture.
Pre- and post-access events are generate for each access. A policy may
have to pre- and/or post-process access, which explains why they may
implement the \code{processAccessRequest(RESTRequestEvent event)} and
\code{processAccessResponse(RESTResponseEvent event)} methods. The
\code{RESTRequestEvent} and \code{RESTResponseEvent} events given as
parameters to these method, contains the information about the access,
its inputs and outputs. The policy may use that information to
determine whether there is a violation, or event modify them, \eg a
paging policy will adapter the request to add page select information.

Each \code{Service} may be initialized with a collection of
\code{UsagePolicy}s or register/unregister policies at runtime.
Policies register at the service repository level are shared by all
services in the repository, \ie any limit enforced by the policies
must be globally satisified by all services. For instance, a shared
policies of $n$ requests per day, will generate an
\code{AccessException} if the limit is reached through any
combinations of accesses.

A service may or may not adhere to shared policy. For a service to adhere to a
policy, the service defintion must referred to the policy by its name
only.
If a policy is fully defined in a single service definition, the scope
of that policy will that service exclusively.
