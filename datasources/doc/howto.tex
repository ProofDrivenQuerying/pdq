\section{How-To\dots}

\subsection{\dots create my own service repository description.}
TODO

\subsection{\dots setup a schema for existing service descriptions.}
Once the service description file is ready, one can refer to the
service defined in it as relations in a schema.

For this, one needs to include the proper set of source descriptions
at the beginning of the schema file.
This is an optional block and was so far used for defining schemas
whose tables could be discovered from a DB.
In this case, ``source'' elements define an external file that contains
description for services. 
Required attributes are ``discoverer'' which here takes the value
``uk.ac.ox.cs.pdq.builder.io.xml.ServiceReader'', and ``file'' which
take the relation path to the service description file.

The schema is defined by listing the relations one wants to include
from those defined in the external service description.
Each service comes with a list of access methods.
One can select in the schema which of those will be used/allowed.
Note that this is why access-methods entries in the schema.xml only
have a name and, optionally, a cost.
The other details about the access-methods are defined in the service
descriptions.

Note that dependencies are defined as before. Here, we have an
inclusion dependency from the latitude and longitude of the Yahoo
service to those of Google.

